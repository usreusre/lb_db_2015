%\documentclass{beamer}
\documentclass[handout]{beamer}


\mode<presentation>
{
  \usetheme{Warsaw}
  \setbeamercovered{transparent}

}


\usepackage[german]{babel}
\usepackage[latin1]{inputenc}


\usepackage{times}
\usepackage[T1]{fontenc}

%f�r Handouts 2 Folien auf einer Din A4 Seite
%\usepackage{pgfpages}
%\pgfpagesuselayout{2 on 1}[a4paper,border shrink=5mm]


\newcommand{\jahr}{2015}
\newcommand{\datum}{21.3.2015}

\title[LB-DB 4] % (optional, nur bei langen Titeln n�tig)
{LB-DB 4 - \datum}

%\subtitle
%{} % (optional)

\author[R. Schlager] % (optional, nur bei vielen Autoren)
{Dipl.-Ing. Reinhard Schlager}
% - Der \inst{?} Befehl sollte nur verwendet werden, wenn die Autoren
%   unterschiedlichen Instituten angeh�ren.

\institute[FH Salzburg] % (optional, aber oft n�tig)
{
%  \inst{1}%
  its\\
  FH Salzburg}
% - Der \inst{?} Befehl sollte nur verwendet werden, wenn die Autoren
%   unterschiedlichen Instituten angeh�ren.
% - Keep it simple, niemand interessiert sich f�r die genau Adresse.

\date[] % (optional)
{\jahr / LB-Datenbanksysteme}


\subject{Informatik}
% Dies wird lediglich in den PDF Informationskatalog einf�gt. Kann gut
% weggelassen werden.


% Falls eine Logodatei namens "university-logo-filename.xxx" vorhanden
% ist, wobei xxx ein von latex bzw. pdflatex lesbares Graphikformat
% ist, so kann man wie folgt ein Logo einf�gen:

\pgfdeclareimage[height=0.5cm]{university-logo}{fh}
\logo{\pgfuseimage{university-logo}}



% Folgendes sollte gel�scht werden, wenn man nicht am Anfang jedes
% Unterabschnitts die Gliederung nochmal sehen m�chte.
\AtBeginSubsection[]
{
  \begin{frame}<beamer>
    \frametitle{Gliederung}
    \tableofcontents[currentsection,currentsubsection]
  \end{frame}
}


% Falls Aufz�hlungen immer schrittweise gezeigt werden sollen, kann
% folgendes Kommando benutzt werden:

\beamerdefaultoverlayspecification{<+->}



\begin{document}

\begin{frame}
  \titlepage
\end{frame}

\begin{frame}
  \frametitle{Gliederung}
  \tableofcontents
  % Die Option [pausesections] k�nnte n�tzlich sein.
\end{frame}


\section{Cursor}

\subsection{Cursor wozu?}
\begin{frame}
\frametitle{Cursor}
\framesubtitle{Warum Cursor?}

\begin{block}{Warum Cursor}
  Zum schrittweisen, prozeduralen Verarbeiten von Records
\end{block}

\begin{block}{Beispiele}
  Laden von Daten (ETL)\\
  Komplexe Reports\\
  Wenn Aufgabe zu komplex f�r ein SQL Statement
\end{block}
\end{frame}

\begin{frame}[fragile]
  \frametitle{Cursor}
  \framesubtitle{Prinzip}
  \begin{semiverbatim}
    \uncover<1->{\alert<1>{DECLARE CURSOR c_name IS}}
    \uncover<1->{\alert<1>{SELECT ... FROM ...; }}
    \uncover<2->{\alert<2>{OPEN c_name;}}    
    \uncover<3->{\alert<3>{FETCH c_name INTO var;}}
  \end{semiverbatim}
\end{frame}

\subsection{Cursor Beispiel}

\begin{frame}[fragile]
  \frametitle{Cursor}
%  \framesubtitle{Beispiel}
\small
\begin{verbatim}
DECLARE
  empId  employees.employee_id%TYPE; 
  fName   employees.first_name%TYPE;
  CURSOR e_curs IS
    SELECT employee_id, first_name 
    FROM employees;
BEGIN
  OPEN e_curs;
  LOOP
    FETCH e_curs INTO empid, fName;
    DBMS_OUTPUT.PUT_LINE('>' || TO_CHAR(empid) || fName);
    EXIT WHEN e_curs%NOTFOUND;
  END LOOP;
  CLOSE e_curs;
END;

\end{verbatim}
\end{frame}

\normalsize



\section{oracle Datentypen}


\begin{frame}
  \frametitle{oracle Datentypen}
  \framesubtitle{}
  \begin{itemize}
\item \emph{VARCHAR2 (n)} - Variable Zeichenkette der maximalen L�nge n,
n zwischen 1 und 4000
\item \emph{VARCHAR (n)} - wie VARCHAR2
\item \emph{CHAR (n)} - Feste Zeichenkette von n Byte, n zwischen 1 und 2000
\item \emph{NCHAR, NVARCHAR} - Zeichenketten mit anderem Zeichensatz als dem der Datenbank
\item \emph{NUMBER (p, s)} - p von 1 bis 38 (Gesamtzahl der Stellen) und s von -84 bis 127 (Vor- bzw. Nachkommastellen)
\item \emph{DATE} - G�ltiger Datumsbereich von -4712 bis 31.12.9999 enth�lt immer auch die sekundengenaue Uhrzeit
  \end{itemize}
\end{frame}


\begin{frame}
  \frametitle{oracle Datentypen}
  \framesubtitle{}
  \begin{itemize}

\item \emph{LONG} - Variable Zeichenkette bis zu 2 GB
\item \emph{RAW (n)} - Bin�rdaten der L�nge n, n zwischen 1 und 2000 Bytes
\item \emph{LONG RAW} - Bin�rdaten bis zu 2 GB
\item \emph{CLOB} - Zeichenketten bis 4 GB
\item \emph{BLOB} - Bin�rdaten bis 4 GB
\item \emph{CFILE, BFILE} - Zeiger auf Dateien (Text, Bin�r)
  \end{itemize}
\end{frame}





\section{Benutzer Berechtigungen}
\subsection{Wof�r?}

\begin{frame}
  \frametitle{oracle Berechtigungen}
  \framesubtitle{Welche Rechte gibt es?}
  \begin{itemize}
  \item SELECT
  \item INSERT
  \item UPDATE
  \item DELETE
  \item ALTER
  \item INDEX
  \item EXECUTE
  \item ...
  \end{itemize}
\end{frame}


\subsection{Syntax}

\begin{frame}[fragile]
  \frametitle{oracle Berechtigungen}
  \framesubtitle{Syntax}
  \begin{semiverbatim}
    \uncover<1->{\alert<0>{GRANT priv\{,priv\}}}
    \uncover<2->{\alert<1>{ON objectname}}
    \uncover<3->{\alert<2>{TO user}}
  \end{semiverbatim}
\end{frame}


\subsection{Beispiel}
\begin{frame}[fragile]
  \frametitle{Oracle Berechtigungen}
  \framesubtitle{Beispiel}
  \begin{semiverbatim}
    \uncover<1->{\alert<0>{GRANT SELECT,DELETE}}
    \uncover<2->{\alert<1>{ON employees}}
    \uncover<3->{\alert<2>{TO db01}}
  \end{semiverbatim}
\end{frame}



\begin{frame}[fragile]
  \frametitle{oracle Berechtigungen entziehen}
  \framesubtitle{}
  \begin{semiverbatim}
    \uncover<0->{\alert<0>{REVOKE priv\{,priv\} ON objectname FROM user}}
    \uncover<1->{\alert<0>{}}
    \uncover<2->{\alert<1>{REVOKE ALL ON employees FROM db01}}
  \end{semiverbatim}
\end{frame}



\begin{frame}[fragile]
  \frametitle{Rollen}
  \framesubtitle{}
  \begin{semiverbatim}
    \uncover<1->{CREATE ROLE emp_readonly}
    \uncover<2->{GRANT SELECT ON employees TO emp_readonly}
    \uncover<3->{GRANT  emp_readonly TO db01}
  \end{semiverbatim}
\end{frame}







\section{�bung 4}

%\begin{frame}
%\frametitle{�bung 4}
%\framesubtitle{4.1 Installateur}

%\begin{block}{Beschreiben Sie folgenden Sachverhalt in einem ER Diagramm}
%Ein Installateur, der sich auf Heizungssysteme spezialisiert hat, betreibt
%Standorte, die teilweise unterschiedlich ausgestattet sind. An jedem Standort
%arbeiten Mitarbeiter mit unterschiedlichen Qualifikationen.Weiters werden
%Kunden sowie deren Heizungssysteme evident gehalten.
%\end{block}

%\begin{block}{}
%Pro Heizungssystem muss festgehalten werden,welcher Leistungstyp 
%( Service, Wechsel von Teilen, ...) an welchem Standort von
%welchen Mitarbeitern erbracht wurden.
%\end{block}
%\end{frame}


\begin{frame}
\frametitle{�bung 4}
\framesubtitle{Archiv}

\begin{block}{Archiv einer Zeitung}
Das Archiv einer Tageszeitung besteht aus mehreren R�umen. In den R�umen
befinden sich Regale mit F�chern. Die Nummer des Fachs ist pro Regal
eindeutig.
In den F�chern sind die einzelnen Ausgaben der Zeitung (identifiziert �ber den
Erscheinungstag) abgelegt.
\end{block}

\begin{block}{}
Jeder Artikel ist einerseits einem Resort (z.B. Politik, Wirtschaft, Sport,
...) andererseits beliebig vielen Schlagw�rtern zugeordnet.
\end{block}

\begin{block}{}
Jeder Artikel wurde von mindestestens einem Journalisten geschrieben.
\end{block}
\end{frame}

\begin{frame}
\frametitle{�bung 4}
\framesubtitle{Archiv}
\begin{block}{Entwickeln Sie ein Schema (ERM $\rightarrow$ Tabellen)}
\dots und f�gen Sie einige geeignete Testdaten in die Tabellen ein
\end{block}

\begin{block}{SQL}
Formulieren Sie ein SQL Statement, dass alle Artikel �ber Datenbanksysteme, 
inkl. der Information wo im Archiv sie zu finden sind, ausgibt.
\end{block}

\end{frame}

\begin{frame}

\begin{block}{stored procedure- freiwillig}
Die Eintragungen bei der Ablage eines Artikels sollen �ber eine stored
procedure gel�st werden.
Schreiben Sie eine geeignete stored procedure, die einen Artikel ablegt
und die entsprechenden Eintragungen in den Tabellen vornimmt.\\
Treffen Sie wenn n�tig Annahmen, aber dokumentieren Sie diese.

\end{block}


\end{frame}


%\begin{frame}
%\frametitle{�bung 4}
%\framesubtitle{Abgabe}

%\begin{block}{elektronisch}
%Zwei ER bis 
%\end{block}


%\end{frame}



\begin{frame}
\frametitle{links}
\begin{enumerate}
\item http://de.wikipedia.org/wiki/SQL
\item http://www.muniqsoft.de/tipps/plsql/index\_tipps-plsql.htm
\item http://www.datenbank-sql.de
\item http://www.torsten-horn.de/techdocs/sql.htm
\item http://www.sql-und-xml.de/xml/sql-tutorial/index
\end{enumerate}
\end{frame}


\end{document}


