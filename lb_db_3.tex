%\documentclass{beamer}
\documentclass[handout]{beamer}


\mode<presentation>
{
  \usetheme{Warsaw}
  \setbeamercovered{transparent}

}


\usepackage[german]{babel}
\usepackage[latin1]{inputenc}
\usepackage{ textcomp }

\usepackage{times}
\usepackage[T1]{fontenc}

%f�r Handouts 2 Folien auf einer Din A4 Seite
%\usepackage{pgfpages}
%\pgfpagesuselayout{2 on 1}[a4paper,border shrink=5mm]

\newcommand{\jahr}{2015}
\newcommand{\datum}{14.3.2015}


\title[LB-DB 3] % (optional, nur bei langen Titeln n�tig)
{LB-DB 3 - \datum}

%\subtitle
%{} % (optional)

\author[R. Schlager] % (optional, nur bei vielen Autoren)
{Dipl.-Ing. Reinhard Schlager}
% - Der \inst{?} Befehl sollte nur verwendet werden, wenn die Autoren
%   unterschiedlichen Instituten angeh�ren.

\institute[FH Salzburg] % (optional, aber oft n�tig)
{
%  \inst{1}%
  its\\
  FH Salzburg}
% - Der \inst{?} Befehl sollte nur verwendet werden, wenn die Autoren
%   unterschiedlichen Instituten angeh�ren.
% - Keep it simple, niemand interessiert sich f�r die genau Adresse.

\date[] % (optional)
{\jahr / LB-Datenbanksysteme}


\subject{Informatik}
% Dies wird lediglich in den PDF Informationskatalog einf�gt. Kann gut
% weggelassen werden.


% Falls eine Logodatei namens "university-logo-filename.xxx" vorhanden
% ist, wobei xxx ein von latex bzw. pdflatex lesbares Graphikformat
% ist, so kann man wie folgt ein Logo einf�gen:

\pgfdeclareimage[height=0.5cm]{university-logo}{fh}
\logo{\pgfuseimage{university-logo}}



% Folgendes sollte gel�scht werden, wenn man nicht am Anfang jedes
% Unterabschnitts die Gliederung nochmal sehen m�chte.
\AtBeginSubsection[]
{
  \begin{frame}<beamer>
    \frametitle{Gliederung}
    \tableofcontents[currentsection,currentsubsection]
  \end{frame}
}


% Falls Aufz�hlungen immer schrittweise gezeigt werden sollen, kann
% folgendes Kommando benutzt werden:

\beamerdefaultoverlayspecification{<+->}



\begin{document}

\begin{frame}
  \titlepage
\end{frame}

\begin{frame}
  \frametitle{Gliederung}
  \tableofcontents
  % Die Option [pausesections] k�nnte n�tzlich sein.
\end{frame}




\section{SQL}
\subsection{Aggregatfunktionen}

\begin{frame}[fragile]
\frametitle{Agregatfunktionen}
\framesubtitle{MAX(),MIN(),SUM(),AVG()}

  \begin{semiverbatim}
    \uncover<1->{\alert<1>{SELECT aggfunction(att1) FROM tab1 [WHERE ...]}}
    \uncover<1->{\alert<1>{}}
    \uncover<2->{\alert<2>{SELECT MAX(salary) FROM employees}} 
    \uncover<3->{\alert<3>{}}
    \uncover<3->{\alert<3>{SELECT AVG(salary) FROM employees }}
    \uncover<3->{\alert<3>{WHERE  department\_id = 100}}
  \end{semiverbatim}
\end{frame}


\subsection{GROUP BY}

\begin{frame}[fragile]
\frametitle{GROUP BY}
\framesubtitle{Gruppieren gleicher Datens�tze}

  \begin{semiverbatim}
    \uncover<1->{\alert<1>{SELECT department\_id,AVG(salary)}}
    \uncover<2->{\alert<2>{FROM employees}} 
    \uncover<3->{\alert<3>{GROUP BY department\_id}}
  \end{semiverbatim}    

\begin{uncoverenv}<4> \footnotesize{
  \begin{tabular}{|l|l|}
    \hline
    DEPARTMENT\_ID & AVG(SALARY)\\
    \hline
    \hline
    100 & 8600\\
    \hline
    30 & 4150\\
    \hline
    ... & ...\\
  \end{tabular}}
\end{uncoverenv}

\end{frame}

\subsection{Sub-SELECT}

\begin{frame}[fragile]
\frametitle{Sub-SELECT}
\framesubtitle{}

  \begin{semiverbatim}
    \uncover<1->{\alert<1>{SELECT e1.department\_id,e1.first\_name,e1.salary}}
    \uncover<2->{\alert<2>{  ,(SELECT avg(e2.salary) FROM employees e2}} 
    \uncover<3->{\alert<3>{    WHERE e1.department\_id = e2.department\_id}}
    \uncover<2->{\alert<2>{    GROUP BY e2.department\_id}}
    \uncover<4->{\alert<4>{    ) AS AVG\_SAL}}
    \uncover<5->{\alert<5>{FROM employees e1}}
  \end{semiverbatim}    

\begin{uncoverenv}<6> \footnotesize{
  \begin{tabular}{|l|l|l|l|}
    \hline 
DEPARTMENT\_ID  &FIRST\_NAME    &SALARY &AVG\_SAL\\
    \hline
    \hline
90    &Steven &24000  &19333.33...\\
    \hline
90      &Neena  &17000  &19333.33...\\
    \hline
90      &Lex    &17000  &19333.33...\\
    \hline
60      &Alexander &&  \\    
... & ...& ... & ...\\ 
  \end{tabular}}
\end{uncoverenv}

\end{frame}

\section{SQL scripts - PL/SQL -}
%\subsection{Beispiel 1}



\begin{frame}[fragile]
  \frametitle{PL/SQL}

  \begin{semiverbatim}
    \uncover<1->{DECLARE}
    \uncover<1->{  m_salary NUMBER(6); nr_days  NUMBER(2);}
    \uncover<1->{  per_day  NUMBER(6,2);}
    \uncover<2->{BEGIN}
    \uncover<2->{  m_salary := 2290;}
    \uncover<2->{  nr_days  := 21;}
    \uncover<2->{  per_day := m_salary/nr_days;}
    \uncover<3->{  DBMS_OUTPUT.PUT_LINE}
    \uncover<3->{       (\textquotesingle per day=\textquotesingle ||TO_CHAR(per_day));}
    \uncover<4->{EXCEPTION}
    \uncover<4->{WHEN ZERO_DIVIDE THEN}
    \uncover<4->{     per_day := 0;}
    \uncover<4->{END;}


  \end{semiverbatim}
\end{frame}


\subsection{stored procedure}

\begin{frame}[fragile]
  \frametitle{stored procedure}
  \begin{semiverbatim}
    \uncover<1->{CREATE PROCEDURE today_is AS}
    \uncover<1->{BEGIN}
    \uncover<1->{  DBMS_OUTPUT.PUT_LINE}
    \uncover<1->{  (\textquotesingle Today is \textquotesingle || TO_CHAR(SYSDATE, \textquotesingle DL \textquotesingle) );}
    \uncover<1->{END today_is;}
    \uncover<2->{---Aufruf durch }
    \uncover<2->{BEGIN}
    \uncover<2->{  today_is();}
    \uncover<2->{END;}

  \end{semiverbatim}
\end{frame}

\subsection{stored function}

\begin{frame}[fragile]
  \frametitle{stored function}
  \begin{semiverbatim}
    \uncover<1->{CREATE FUNCTION worked_for (empid NUMBER)}
    \uncover<1->{ RETURN VARCHAR2 IS}
    \uncover<2->{ years INT;}
    \uncover<3->{BEGIN}    
    \uncover<4->{ SELECT round((sysdate-hire_date )/365)}
    \uncover<4->{ INTO years FROM employees}
    \uncover<4->{ WHERE employee_id = empid;}
    \uncover<5->{ RETURN (\textquotesingle worked for approx. \textquotesingle}
    \uncover<5->{ || years ||\textquotesingle years\textquotesingle);}
    \uncover<3->{END worked_for;}

  \end{semiverbatim}
\end{frame}

\begin{frame}[fragile]
  \frametitle{Anwendung}
  \begin{semiverbatim}
    \uncover<1->{SELECT hire_date,worked_for(employee_id)}
    \uncover<1->{FROM employees}
    \uncover<1->{ORDER BY hire_date}

  \end{semiverbatim}
\end{frame}


%einauen
%declare status integer;
%begin
%    status:= transfer(1,2,1000)
%end;

\subsection{Trigger}
\begin{frame}[fragile]
  \frametitle{Trigger}
  \begin{semiverbatim}
    \uncover<1->{CREATE OR REPLACE TRIGGER audit_sal}
    \uncover<2->{AFTER UPDATE OF salary}
    \uncover<3->{ON employees FOR EACH ROW}
    \uncover<4->{BEGIN}
    \uncover<5->{INSERT INTO emp_audit VALUES}
    \uncover<5->{( :OLD.employee_id, SYSDATE,} 
    \uncover<5->{  :NEW.salary, :OLD.salary );}
    \uncover<4->{END;}

  \end{semiverbatim}
\end{frame}


\subsection{sequence}
\begin{frame}[fragile]
  \frametitle{sequence}
  \begin{semiverbatim}
    \uncover<1->{CREATE SEQUENCE new_employees_seq}
    \uncover<2->{START WITH 1000 INCREMENT BY 1;}
    \uncover<3->{}
    \uncover<3->{INSERT INTO employees}
    \uncover<3->{(employee_id,first_name,}
    \uncover<3->{last_name,email,hire_date,job_id)}
    \uncover<3->{VALUES} 
    \uncover<4->{\alert{(new_employees_seq.nextval,}}
    \uncover<4->{\textquotesingle a\textquotesingle,\textquotesingle b\textquotesingle,\textquotesingle c\textquotesingle,\textquotesingle 14-mar-2015\textquotesingle,\textquotesingle SA_MAN\textquotesingle)}
  \end{semiverbatim}
\end{frame}

\subsection{index}

\begin{frame}[fragile]
  \frametitle{CREATE INDEX}

  \begin{semiverbatim}
    \uncover<1->{CREATE [UNIQUE] INDEX <index_name>}
    \uncover<2->{ON <table_name>}
    \uncover<3->{(<field_name>\{<field_name>\})}
  \end{semiverbatim}
\end{frame}

\begin{frame}[fragile]
  \frametitle{Beispiel}

  \begin{semiverbatim}
    \uncover<1->{CREATE UNIQUE INDEX emp_mgr_id_ix}
    \uncover<2->{ON employees}
    \uncover<3->{(employee_id)}
  \end{semiverbatim}
\end{frame}


\section{�bung}
\begin{frame}
\frametitle{�bung 3 LB-DB \datum}
\framesubtitle{}
\begin{itemize}
\item Schreiben Sie eine Funktion dif\_to\_avg(employee\_id), die die Abweichung
  des Gehalts des Mitarbeiters vom Durchschnitt ermittelt.
\item Schreiben Sie einen Trigger, der das Reduzieren eines Gehalts verhindert.
  Wenn der neue Gehalt kleiner als der alte Gehalt ist, soll der Gehalt nicht ver�ndert werden.
\item Formulieren Sie 3 Abfrage auf Basis des HR Schemas. In diesen Abfragen
m�ssen die Konstrukte GROUP BY, HAVING, MAX zumindest einmal
vorkommen. Beschreiben Sie die Abfrage und zeigen Sie das Ergebnis.
\end{itemize}
\end{frame}

%\section{Links}

%\begin{frame}
%\frametitle{Links}
%\url{http://de.wikipedia.org/wiki/PL/SQL}
%\end{frame}


\end{document}

