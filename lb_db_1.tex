%\documentclass{beamer}
\documentclass[handout]{beamer}


\mode<presentation>
{
  \usetheme{Warsaw}
  \setbeamercovered{transparent}

}


\usepackage[german]{babel}
\usepackage[latin1]{inputenc}


\usepackage{times}
\usepackage[T1]{fontenc}


%f�r Handouts 2 Folien auf einer Din A4 Seite
%\usepackage{pgfpages}
%\pgfpagesuselayout{2 on 1}[a4paper,border shrink=5mm]

\newcommand{\jahr}{2015}
\newcommand{\datum}{28.2.2015}


\title[LB-DB 1] % (optional, nur bei langen Titeln n�tig)
{LB-DB 1  \datum}

%\subtitle
%{} % (optional)

\author[R. Schlager] % (optional, nur bei vielen Autoren)
{Dipl.-Ing. Reinhard Schlager}
% - Der \inst{?} Befehl sollte nur verwendet werden, wenn die Autoren
%   unterschiedlichen Instituten angeh�ren.

\institute[FH Salzburg] % (optional, aber oft n�tig)
{
%  \inst{1}%
  its\\
  FH Salzburg}
% - Der \inst{?} Befehl sollte nur verwendet werden, wenn die Autoren
%   unterschiedlichen Instituten angeh�ren.
% - Keep it simple, niemand interessiert sich f�r die genau Adresse.

\date[] % (optional)
{\jahr / LB-Datenbanksysteme}


\subject{Informatik}
% Dies wird lediglich in den PDF Informationskatalog einf�gt. Kann gut
% weggelassen werden.


% Falls eine Logodatei namens "university-logo-filename.xxx" vorhanden
% ist, wobei xxx ein von latex bzw. pdflatex lesbares Graphikformat
% ist, so kann man wie folgt ein Logo einf�gen:

\pgfdeclareimage[height=0.5cm]{university-logo}{fh}
\logo{\pgfuseimage{university-logo}}



% Folgendes sollte gel�scht werden, wenn man nicht am Anfang jedes
% Unterabschnitts die Gliederung nochmal sehen m�chte.
\AtBeginSubsection[]
{
  \begin{frame}<beamer>
    \frametitle{Gliederung}
    \tableofcontents[currentsection,currentsubsection]
  \end{frame}
}


% Falls Aufz�hlungen immer schrittweise gezeigt werden sollen, kann
% folgendes Kommando benutzt werden:

\beamerdefaultoverlayspecification{<+->}



\begin{document}

\begin{frame}
  \titlepage
\end{frame}

\begin{frame}
  \frametitle{Gliederung}
  \tableofcontents
  % Die Option [pausesections] k�nnte n�tzlich sein.
\end{frame}

\section{Anmeldung}

\begin{frame}
\frametitle{oracle}
\begin{itemize}
   \item https://iacademy.oracle.com
   \item Workspace AT\_1532
   \item Username AT\_1532\_SQL01\_Snn
\end{itemize}
\end{frame}




\begin{frame}
\frametitle{}
\includegraphics<1>[height=8cm]{../pics/login.pdf}
\end{frame}


\begin{frame}
\frametitle{}
\includegraphics<1>[height=8cm]{../pics/menu1.pdf}
\end{frame}

\begin{frame}
\frametitle{}
\includegraphics<1>[height=8cm]{../pics/menu2.pdf}
\end{frame}

\begin{frame}
\frametitle{}
\includegraphics<1>[height=8cm]{../pics/menu3.pdf}
\end{frame}

\section{SQL �bersicht}

\begin{frame}
\frametitle{SQL �bersicht}

\begin{itemize}
\item Data Definition Language (DDL)
  \begin{itemize}
  \item CREATE
  \item ALTER
  \item DROP
  \end{itemize}
\item Data Manipulation Language (DML)
  \begin{itemize}
  \item INSERT
  \item UPDATE
  \item DELETE
  \end{itemize}
\item SELECT
\end{itemize}

\end{frame}


\subsection{CREATE TABLE}
\begin{frame}[fragile]
\frametitle{CREATE TABLE}
\begin{semiverbatim}
\uncover<1->{\alert<1>{CREATE TABLE tabellenname}}
\uncover<2->{\alert<2>{(feldname datentyp\{,feldname datentyp\}}}
\uncover<2->{\alert<2>{)}}
\uncover<2->{\alert<2>{ }}
\uncover<3->{\alert<3>{CREATE TABLE person}}
\uncover<4->{\alert<4>{(persnr INT}}
\uncover<4->{\alert<4>{,vorname  CHAR(45)}}
\uncover<4->{\alert<4>{,nachname CHAR(45)}}
\uncover<4->{\alert<4>{)}}
\end{semiverbatim}
\end{frame}


\begin{frame}
\frametitle{}
\includegraphics<1>[height=8cm]{../pics/createtable.pdf}
\end{frame}

\begin{frame}
\frametitle{}
\includegraphics<1>[height=8cm]{../pics/describe.pdf}
\end{frame}
\subsection{INSERT}

\begin{frame}[fragile]
\frametitle{INSERT}
\framesubtitle{Syntax}
\begin{semiverbatim}
\uncover<0->{\alert<0>{INSERT INTO tabellename}}
\uncover<1->{\alert<1>{(spalte\{,spalte\})}}
\uncover<2->{\alert<2>{VALUES}}
\uncover<3->{\alert<3>{(wert \{,wert\})}}
\end{semiverbatim}
\end{frame}

\begin{frame}[fragile]
\frametitle{INSERT}
\framesubtitle{Beispiel}
\begin{semiverbatim}
\uncover<0->{\alert<0>{INSERT INTO person}}
\uncover<1->{\alert<1>{(persnr,vorname,nachname)}}
\uncover<2->{\alert<2>{VALUES}}
\uncover<3->{\alert<3>{(1,'Reinhard','Schlager')}}
\end{semiverbatim}
\end{frame}

\subsection{SELECT}
\begin{frame}[fragile]
\frametitle{SELECT}
\framesubtitle{Syntax}
\begin{semiverbatim}
\uncover<0->{\alert<0>{SELECT}}
\uncover<1->{\alert<1>{\{*|spalte|ausdruck\}}}
\uncover<2->{\alert<2>{FROM tabellenname}}
\uncover<3->{\alert<3>{[WHERE bedingung]}}
\uncover<4->{\alert<4>{[ORDER BY \{spalte|ausdruck\} [ASC|DESC] }}
\end{semiverbatim}
%\visible<4->{Note the use of \alert{\texttt{std::}}.}
\end{frame}

\begin{frame}[fragile]
\frametitle{SELECT}
\framesubtitle{Beispiel}
\begin{semiverbatim}
\uncover<0->{\alert<0>{SELECT}}
\uncover<1->{\alert<1>{vorname,nachname}}
\uncover<2->{\alert<2>{FROM person}}
\uncover<3->{\alert<3>{WHERE nachname LIKE 'M%'}}
\uncover<4->{\alert<4>{ORDER BY persnr}}
\end{semiverbatim}
%\visible<4->{Note the use of \alert{\texttt{std::}}.}
\end{frame}

\section{�bung 1}


\begin{frame}
\frametitle{�bung 1 LB-DB \datum}
\framesubtitle{Abgabe (Papier) und Pr�sentation in der n�chsten �bung}
\begin{enumerate}
\item Suchen Sie die Online Dokumentation zu ORACLE SQL
\item Entwerfen Sie das Datenmodell eines einfachen Fahrplans.
  (Bahnh�fe, Z�ge, Abfahrten, Ank�nfte)
\item Beschreiben Sie in welcher Form Sie die Daten abspeichern wollen
\item Formulieren Sie 3 sinnvolle Anfragen an Ihr System als Text.
\item Schreiben Sie geeignete SQL-DDL Statements, die die entsprechenden Tabellen erzeugen und protokollieren Sie den Ablauf.
\end{enumerate}
\end{frame}

\begin{frame}
\frametitle{�bung 1(2) LB-DB \datum}
\framesubtitle{Abgabe (Papier) und Pr�sentation in der n�chsten �bung}
\begin{enumerate}
\setcounter{enumi}{5}
\item Bef�llen Sie die Tabellen mit einigen sinnvollen Datens�tzen durch die entsprechenden SQL Befehle und protokollieren Sie den Ablauf.
\item Versuchen Sie jetzt, die 3 Abfragen aus Aufgabe 4 als SQL Kommandos zu
  formulieren. Protokollieren Sie die Kommandos und das Ergebnis.
\item Freiwillig - Wie w�rden Sie Ihr Datenmodell grafisch darstellen wollen.
\end{enumerate}
\end{frame}


\subsection{Links}


\begin{frame}
\frametitle{links}
\begin{enumerate}
%\item http://www.users.fh-salzburg.ac.at/\%7Enulamec
\item http://de.wikipedia.org/wiki/SQL
\item http://www.sql-und-xml.de/xml/sql-tutorial/index
\end{enumerate}
\end{frame}




\end{document}


